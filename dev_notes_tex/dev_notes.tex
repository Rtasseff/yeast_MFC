% latex file for generating the development notes

\documentclass{article}

\usepackage{cite}% Make references as [1-4], not [1,2,3,4]
\usepackage{url}  % Formatting web addresses 
\usepackage{amsmath}
\usepackage{amssymb}
\usepackage{epsfig}


% Article top matter
\title{yeast\_MFC Developer Notes} %\LaTeX is a macro for printing the Latex logo
\author{Ryan Tasseff, \texttt{rtasseff@systemsbiolgy.org}}  %\texttt formats the text to a typewriter style font
\date{\today}  %\today is replaced with the current date



\begin{document}

\maketitle


\begin{abstract}
Detailed notes of the development of the yeast microfludic chip model,
using the Biocellion framework.
The model will capture the behavior of 
budding yeast and the surrounding environment in a chip designed 
designed to trap single yeast cells for time laps imaging.
Here we track the development of the model over time.
A loose organizational scheme is used, with components then function then time.
The document is dynamic will will change and the project progresses. \end{abstract}
 
\section{Introduction}
Here we will cover the development of the yeast microfludic chip model,
using the Biocellion framework.
The primary goal of the model is to capture the behavior of 
budding yeast and the surrounding environment in a chip  
designed to trap single yeast cells for time laps imaging.
Specifically, we wish to model simple aspect of yeast: 
movement, budding/ division, death and nutrient uptake.
We also want to capture the diffusion and (if needed) the 
convection/ advection of key molecules, 
to model there spatial distribution over time.

The following is a rough guide to how the model was/ is being developed.
There is no specific structure; however, 
we will attempt to organize first by specific model components, 
then by relevant biology/ function and finally by date (time stamp on all additions).
To some extent, we will leave deprecated code/ descriptions
in the notes; however, we will try to mark them as deprecated.

\section{General Assumptions, Simplifications, Details}
\subsection{Growth and Division}
\emph{20131111} We are assuming a simple model of growth, division and cell cycle control\cite{Charvin2009}.
The primary assumptions are 1) exponential growth of cells
2) perfect volume control,
the cells do not start a clock until some $V_C$ is reached;
2b) all additional volume will be transfered to daughter;
3) linear progression through cell cycle;
3b) completely deterministic mitosis, 
the cell finishes division after clock hits 1.
Currently, we have added some additional simplification and assumptions:
1) cells do not bud they just grow and divide end of cycle,
this will be improved by forming a junction with m and d cells after g1, 
which will break after cycle ends,
unfortunately it may be trick to implement exp vol increase in d based on m vol;
2) budding is always at same spot so we will track it,
unfortunately this will not do much until we implement rotation

\subsection{Physical Environment}
\emph{20131111} Although the model does contain 3 dimensions,
we will focus only on the first 2 (x,y), 
and assume movment is restricted in z-axis based on chip size.

\section{Initialization}
\emph{20130927} We have initialized the model using the PDE example provided in Biocellion.

\section{Model Properties and Definitions}
The following is a general overview of the various elements on the model.
Nearly all is defined in model \emph{model\_define.h}.

\subsection{Agent types and properties}
\emph{20130927} We will add agents under `---Agents---'.
The primary agent is \texttt{AGENT\_YEAST\_CELL}.
We also added the initial properties in the same place,
specifically we have added 2 model reals that describe the 
location of the budding direction \texttt{YEAST\_CELL\_MODEL\_REAL\_BUD\_DIR\_}.
\emph{20130927} The property \texttt{YEAST\_CELL\_MODEL\_REAL\_CC\_CLOCK}
was added to monitor progression through the cell cycle \cite{Charvin2009}
Additional cell properties added under `---Cell Properties--- including 
`-Growth and Division-' properties taken from \cite{Charvin2009} to describe the cell cycle.

\section{Agent Rules}
\emph{20131111} Rules defined in \emph{model\_routine\_agent.cpp} unless otherwise stated.

\subsection{updateSpAgent}
\emph{20131111} We define growth here.  
Currently, we have deterministic exponential volume increase.
Which is achieved by simple steps.  
We have added the growth here as we may wish to couple it to the environment via uptake.

\subsection{updateSpAgentBirthDeath }
\emph{20131111} We are doing all division at a single step,
as opposed to a bud.
So when cell cycle is complete we allow the cells to divide.
Based on \cite{Charvin2009} cell cycle is complete after the clock reaches 1.0.
The limit is defined in \emph{model\_define.h} as \texttt{CC\_CLOCK\_CRITICAL}.
We note that the phase is tracked as an internal variable \texttt{YEAST\_CELL\_MODEL\_REAL\_CC\_CLOCK}

\subsection{adjustSpAgentState}
\emph{20131111} Here have track movement through the cell cycle by linearly increasing the phase varriable under ` ---Cell Cycle Progression---'.
We have done this here as it only leads to division, and division is done at the baseline time step, as this module is.

\subsection{divideSpAgent}
\emph{20131111} Division is done all at once.  
After the cell cycle is complete this method will be called.
All volume above the critical volume is sent to the daughter.
Displacment is done so that both move to account for new radius, 
and this is in the direction of the budding, defined as a property in \emph{model\_define.h}.
 

 





\end{document}  %End of document.
