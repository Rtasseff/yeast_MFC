% latex file for generating the development notes

\documentclass{article}

\usepackage{cite}% Make references as [1-4], not [1,2,3,4]
\usepackage{url}  % Formatting web addresses 
\usepackage{amsmath}
\usepackage{amssymb}
\usepackage{epsfig}


% Article top matter
\title{yease\_MFC Developer Notes} %\LaTeX is a macro for printing the Latex logo
\author{Ryan Tasseff, \texttt{rtasseff@systemsbiolgy.org}}  %\texttt formats the text to a typewriter style font
\date{\today}  %\today is replaced with the current date



\begin{document}

\maketitle


\begin{abstract}
Detailed notes of the development of the yeast microfludic chip model,
using the Biocellion framework.
The model will capture the behavior of 
budding yeast and the surrounding environment in a chip designed 
designed to trap single yeast cells for time laps imaging.
Here we track the development of the model over time.
A loose organizational scheme is used, with components then function then time.
The document is dynamic will will change and the project progresses. \end{abstract}
 
\section{Introduction}
Here we will cover the development of the yeast microfludic chip model,
using the Biocellion framework.
The primary goal of the model is to capture the behavior of 
budding yeast and the surrounding environment in a chip designed 
designed to trap single yeast cells for time laps imaging.
Specifically, we wish to model simple aspect of yeast: 
movement, budding/ division, death and nutrient uptake.
We also want to capture the diffusion and (if needed) the 
convection/ advection of key molecule, 
to model there spatial distribution over time.

The following is a rough guide to how the model was/ is being developed.
There is no specific structure; however, 
we will attempt to organize first by specific model components, 
then by relevant biology/ function and finally by date (time stamp on all additions).
To some extent, we will leave deprecated code/ descriptions
in the notes; however, we will try to mark them as deprecated.


\end{document}  %End of document.
